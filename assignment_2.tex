%----------------------------------------------------------------------------------------
%	PACKAGES AND DOCUMENT CONFIGURATIONS
%----------------------------------------------------------------------------------------

\documentclass{article}

\usepackage{graphicx} %For Images
%\usepackage{amsmath} If we need math

\setlength\parindent{0pt}

\renewcommand{\labelenumi}{\alph{enumi}.}

%----------------------------------------------------------------------------------------
%	DOCUMENT INFORMATION
%----------------------------------------------------------------------------------------


\author{
	Mahmoud, Youssef\\
	\texttt{yam160130@utdallas.edu}
	\and
	Harris, Collin\\
	\texttt{chh150030@utdallas.edu}
	\and
	Park, Daniel\\
	\texttt{dxp123230@utdallas.edu}
	\and
	Foster, Joseph\\
	\texttt{jpf150030@utdallas.edu}
}
\date{October 24, 2019}
\title{
  \textbf{
    Development of a Simple \\ Train Controller \\ CS 4397
  }
}

%----------------------------------------------------------------------------------------
%	DOCUMENT CONTENT
%----------------------------------------------------------------------------------------

\begin{document}
  \maketitle

  \section{Objective}
    To develop a program for controlling the train in Lab ECSS 3.217. The control system
		should implement a simple user interface that allows the user to \textbf{ring} the
		bell, \textbf{start} the train, \textbf{accelerate} the train, \textbf{move} the
		train, \textbf{decelerate} the train, and \textbf{stop} the train.

  \section{Procedure}
	  One of the main features that is unique with our program is the use of macros. This is due to being limited to
	  C, which places restrictions on how constants can be used. The program works by taking user input, then sending messages
	  over UART to the train system. The program is basically a relay system between the user and the train. It take in a command 
	  from the user and sends the corresponding UART message to the train. The main differance between this program and general purpose programs is that this one controls hardware. That makes this an embedded systems program.

\end{document}
